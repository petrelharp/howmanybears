\RequirePackage{fix-cm}
\documentclass[12pt]{article}
\usepackage[fontsize=12pt]{scrextend}
\usepackage{wrapfig}
\usepackage{geometry} 
\usepackage{graphicx}
\usepackage{float}
\usepackage{url}
\usepackage{mathtools}
\usepackage{amsmath}
\usepackage{amsfonts}
\usepackage{bigints}
\usepackage{soul,color}
\usepackage{epigraph}
\usepackage[super,numbers,sort&compress]{natbib}
\usepackage[font=scriptsize,labelfont=bf]{caption}
\usepackage[parfill]{parskip}
\parskip=8pt
\pagenumbering{gobble}

\bibliographystyle{plos2015}

\geometry{
a4paper,
top=1in,
left=1in,
right=1in,
bottom=1in
} 

%	 NOTES ON PROPOSAL
%            1-2pg pre-proposal that includes:
%		general budget
%		relevant personnel (Peter, Gideon, a pdoc tbd)
%		approach
%		deliverables (density, movement, xvalidation w/ previous years/findings)
%			timeline
%		dataset will consist of ~150-200 SNPs


\begin{document}
%
\begin{center}
\textbf{Developing a close-kin mark-recapture model to map black bear population numbers in Michigan (Upper Peninsula)}
\end{center}
%            
\begin{itemize}
\item motivation: what are we trying to do and why

Getting accurate population size estimates 
for managed or threatened species  
is vital for implementing informed strategies for managing them.
In Michigan, black bears (\textit{Ursus americanus}) 
are managed by harvesting, 
and the hunting license quota is determined by 
estimates of their population size.

\item brief description of current methods (CMR)

Recent bear population monitoring efforts 
in the state of Michigan have relied on 
capture-mark-recapture (CMR) methods, 
by which a sample of a population is initially marked, 
and an estimate of total population abundance 
can be made from the number of marked individuals 
that are subsequently recaptured (or harvested). 

\item areas for improvement

However, traditional CMR methods are not ideal;  
they require extra fieldwork to do initial population marking, 
the tetracycline biomarker used in marking is no longer permitted, 
and there is a substantial wait time (1-3 years) after the 
initial marking period until population estimates can be made.
Advances in CMR methods, 
such as statistical catch-at-age analyses (SCAA),
address some of these weaknesses, 
but crucially still require a costly and time-intensive 
marking field effort.

\item ``we propose...'' 

Using genetic data and patterns of relatedness 
between censused individuals offers 
an exciting and cost-effective alternative to CMR.

\item credentials/expertise 
\begin{itemize}
\item spatial popgen
\item statistical methods
\item previous bear research
\item deliverables for informing government action (torts)
\end{itemize}
\item description of deliverables (explicit about what we could offer over existing methods)
\begin{itemize}
\item map of density
\item accompanying map of uncertainty
\item maybe map of dispersal
\item highlight areas w/ outlier dispersal
\item might reveal MUs for harvesting
\item some validation and caveats
\item user interface and accompanying training 
\end{itemize}
\item budget: pdoc and summer salary for PIs for 2 yrs
\item timeline (gant chart?)
\begin{itemize}
\item hire personnel
\item devleop \& test methods
\item get UI up and running
\item do preliminary trainings
\item tweak
\item do more trainings \& release
\end{itemize}
%look deeper in pedigree
% value added: all inds simultaneously vs. one dyad at a time?
%MI-specifc simulation
\end{itemize}

\clearpage
%\bibliography{../../../references/reference.library.bib}
\end{document}